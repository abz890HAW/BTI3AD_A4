%----------------------------------------------------------------------------------------
% PACKAGES AND OTHER DOCUMENT CONFIGURATIONS
%----------------------------------------------------------------------------------------

\documentclass[paper=a4, fontsize=11pt]{scrartcl} % A4 paper and 11pt font size

\usepackage[margin=1in]{geometry}
\usepackage[T1]{fontenc} % Use 8-bit encoding that has 256 glyphs
\usepackage{fourier} % Use the Adobe Utopia font for the document - comment this line to return to the LaTeX default
\usepackage[ngerman]{babel} % German language/hyphenation
\usepackage[utf8]{inputenc}
\usepackage{amsmath,amsfonts,amsthm} % Math packages
\usepackage{booktabs}
\usepackage{sectsty} % Allows customizing section commands
\usepackage{listings}
\usepackage{xcolor}
\usepackage{pgfplotstable}
\usepackage{pgfplots}

\allsectionsfont{\centering \normalfont\scshape} % Make all sections centered, the default font and small caps

\usepackage{fancyhdr} % Custom headers and footers
\pagestyle{fancyplain} % Makes all pages in the document conform to the custom headers and footers
\fancyhead{} % No page header - if you want one, create it in the same way as the footers below
\fancyfoot[L]{} % Empty left footer
\fancyfoot[C]{} % Empty center footer
\fancyfoot[R]{\thepage} % Page numbering for right footer
\renewcommand{\headrulewidth}{0pt} % Remove header underlines
\renewcommand{\footrulewidth}{0pt} % Remove footer underlines
\setlength{\headheight}{13.6pt} % Customize the height of the header

\numberwithin{equation}{section} % Number equations within sections (i.e. 1.1, 1.2, 2.1, 2.2 instead of 1, 2, 3, 4)
\numberwithin{figure}{section} % Number figures within sections (i.e. 1.1, 1.2, 2.1, 2.2 instead of 1, 2, 3, 4)
\numberwithin{table}{section} % Number tables within sections (i.e. 1.1, 1.2, 2.1, 2.2 instead of 1, 2, 3, 4)

\setlength\parindent{0pt} % Removes all indentation from paragraphs - comment this line for an assignment with lots of text

%----------------------------------------------------------------------------------------
% TITLE SECTION
%----------------------------------------------------------------------------------------

\newcommand{\horrule}[1]{\rule{\linewidth}{#1}} % Create horizontal rule command with 1 argument of height

\title{
\normalfont \normalsize
\textsc{Hochschule für Angewandte Wissenschaften Hamburg, Department Informatik} \\ [25pt] % Your university, school and/or department name(s)
\horrule{0.5pt} \\[0.4cm] % Thin top horizontal rule
\huge Algorithmen und Datenstrukturen - Aufgabe 4 \\ % The assignment title
\horrule{2pt} \\[0.5cm] % Thick bottom horizontal rule
}

\author{Hauke Goldhammer, Markus Blechschmidt} % Your name

\date{\normalsize\today} % Today's date or a custom date

\lstdefinestyle{customjava}{
  belowcaptionskip=1\baselineskip,
  breaklines=true,
  frame=single,
  captionpos=b,
  xleftmargin=\parindent,
  language=Java,
  showstringspaces=false,
  keepspaces=true,
  basicstyle=\footnotesize\ttfamily,
  keywordstyle=\bfseries\color{green!40!black},
  commentstyle=\itshape\color{purple!40!black},
  identifierstyle=\color{blue},
  stringstyle=\color{orange},
  numberstyle=\tiny\color{gray},
  numbers=left,                    % where to put the line-numbers; possible values are (none, left, right)
  numbersep=5pt,                   % how far the line-numbers are from the code
}
\lstset{style=customjava}

\begin{document}


\maketitle % Print the title

\subsubsection*{Abstract}
In dieser Aufgabe soll sich mit rekursiven und iterativen Verfahren am Beispiel
des Pascalschen Dreiecks beschäftigt werden. Dieses Dokument stellt einen L\"osungsvorschlag
dar und beinhaltet die Aufbereitung und heuristische Auswertung der Schrittz\"ahler
in den Implementationen, so wie eine analytische Herleitung.

\renewcommand{\contentsname}{Inhaltsangabe}
\renewcommand*\listtablename{Tabellen}
\renewcommand*\listfigurename{Abbildungen}
\renewcommand{\figurename}{Abb.}
\tableofcontents
%\listoftables
\listoffigures
%\lstlistoflistings

\clearpage

\section{Algorithmen}

Dieser Abschnitt beschreibt grundlegend die Funktion und Implementation der Algorithmen.
Alle Algorithmen implementieren das gleiche Interface. Dieses beschreibt, dass die
Funktionen als einzigen Parameter entgegennimmt, welche Zeile errechnet werden soll.
Der R\"uckgabetyp ist ein Zahlenfeld, welches die entsprechenden Zeile darstellt.

\subsection{Rekursiv}
Die Abbruchbedienung f\"ur diese Implementation ist, dass man die 1. oder eine
``kleine'' Zeile errechnen l\"asst. In diesem Fall wird $[1, 1]$ zur\"uckgegeben.
In alles anderen F\"allen wird erst die vorhergehende Zeile errechnet und dann
damit iterativ die Elemente der n\"achste Zeile generiert.

\subsection{Iterativ}
Dieser Algorithmus generiert als erstes die erste Zeile. Danach werden nach und
nach darauf basierend alle Weiteren Zeilen aus der vorhergehenden generiert, bis
man die gew\"unschte Zeile erreicht hat.

\subsection{Direkt}
Dieser Algorithmus generiert alle Elemente der Zeile mit dem Binominialkoeffizienten
$\binom{n}{k}$, wobei $n$ die Zeilennummer ist und $k$ die Position in der Zeile.

\section{Laufzeitergebnisse}

Die Abbildung \ref{figure:loglog} zeigt grafisch den Zusammenhang zwischen
Problemgr\"o{\ss}e $N$, welche die Nummer der Reihe des Pascalschen Dreiecks darstellt,
welche errechnet werden soll, und $T(N)$, welches das Ergebnis unseres eingebauten
Schrittz\"ahlers ist.

Bereits an Abbildung \ref{figure:loglog} ist zu erkennen, dass der iterative und rekursive
Algorithmus ein st\"arkeres Aufwandswachstum aufweisen als der direkte Algorithmus.
Dies zeigt sich auch noch mal deutlich in der Abbildung \ref{figure:loglogeq},
in welcher beide Achsen des Koordinatensystems gleich skaliert sind. Hieran kann
man \"uber die Steigung auch ablesen, dass der interative und rekursive Algorithmus
ein quadratisches Wachstum aufweisen (Steigung 2) und der direkte Algorithmus
ein lineares (Steigung 1).

\begin{figure}[h]
  \centering
  \begin{tikzpicture}
    \begin{loglogaxis}[
      xlabel={Problemgr\"o{\ss}e $N$},
      ylabel={Aufwand $T(N)$},
      xmin=1,
      ymin=1,
      legend pos=north west,
      xmajorgrids=true,
      ymajorgrids=true,
      grid style=dashed,
      % axis equal,
      ]
      \addplot +[mark size=0.5pt] table [x=N, y=reku] {dat/benchmark.dat};
      \addplot +[mark size=0.5pt] table [x=N, y=ite, mark=none] {dat/benchmark.dat};
      \addplot +[mark size=0.5pt] table [x=N, y=fast, mark=none] {dat/benchmark.dat};
      \legend{$T(N)_{rekursiv}$,$T(N)_{iterativ}$,$T(N)_{direkt}$}
    \end{loglogaxis}
  \end{tikzpicture}
  \caption{Graph des Aufwands $T(N)$ f\"ur alle drei Funktionen bei steigender Aufwandsgr\"o{\ss}e $N$ in einer log-log-Skala.}
  \label{figure:loglog}
\end{figure}

\begin{figure}[h]
  \centering
  \begin{tikzpicture}
    \begin{loglogaxis}[
      xlabel={Problemgr\"o{\ss}e $N$},
      ylabel={Aufwand $T(N)$},
      xmin=1,
      ymin=1,
      legend pos=north east,
      xmajorgrids=true,
      ymajorgrids=true,
      grid style=dashed,
      axis equal,
      ]
      \addplot +[mark size=0.5pt] table [x=N, y=reku] {dat/benchmark.dat};
      \addplot +[mark size=0.5pt] table [x=N, y=ite, mark=none] {dat/benchmark.dat};
      \addplot +[mark size=0.5pt] table [x=N, y=fast, mark=none] {dat/benchmark.dat};
      \legend{$T(N)_{rekursiv}$,$T(N)_{iterativ}$,$T(N)_{direkt}$}
    \end{loglogaxis}
  \end{tikzpicture}
  \caption{Graph des Aufwands $T(N)$ f\"ur alle drei Funktionen bei steigender Aufwandsgr\"o{\ss}e $N$ in einer symmetrischen log-log-Skala.}
  \label{figure:loglogeq}
\end{figure}

Die quadratischen Laufzeiten der iterativen und rekursiven L\"osung lassen sich
mit der Gau{\ss}schen Summenformel erkl\"aren. F\"ur jede Zeile braucht man die
Zeile davor und jede Zeile braucht einen Rechenschritt mehr als die davor.

Bzw.:
\begin{equation}
  \begin{split}
    T(1)_{rekursiv} &= 0 \\
    T(N)_{rekursiv} &= T(N-1)_{rekursiv} + N - 1 = \mathcal{O}(n^2)\\
  \end{split}
\end{equation}

\begin{equation}
  \begin{split}
    T(N)_{iterativ} &= \sum_{i=0}^{N-1} i = \mathcal{O}(n^2)\\
  \end{split}
\end{equation}

\begin{equation}
  \begin{split}
    T(N)_{direkt} &= \left(\sum_{i=0}^{N-1} 1\right) + \left(\sum_{i=0}^{N-1} 1\right) = \sum_{i=0}^{N-1} 2 = \mathcal{O}(n)\\
  \end{split}
\end{equation}


\end{document}
